\documentclass[aspectratio=169]{beamer}
\usetheme{Madrid}
\usecolortheme{default}

\usepackage{amsmath}
\usepackage{amssymb}
\usepackage{graphicx}
\usepackage{booktabs}
\usepackage{tikz}
\usepackage{hyperref}

\title{Chaos in Hyperdimensional Rubik's Cubes}
\subtitle{Discrete Dynamical Systems on 4D Puzzles}
\author{Math 538 Final Project}
\date{\today}

\begin{document}

% Title slide
\begin{frame}
\titlepage
\end{frame}

% Outline
\begin{frame}{Overview}
\tableofcontents
\end{frame}

\section{Introduction}

\begin{frame}{What is a 4D Rubik's Cube?}
\begin{columns}
\column{0.5\textwidth}
\textbf{3D Rubik's Cube:}
\begin{itemize}
    \item 3×3×3 puzzle
    \item 6 faces (F, U, R, L, B, D)
    \item $\sim 4.3 \times 10^{19}$ states
    \item Rotations in 3D space
\end{itemize}

\vspace{1em}

\textbf{4D Hypercube:}
\begin{itemize}
    \item 3×3×3×3 puzzle
    \item \textcolor{red}{8 cells} (3D "faces")
    \item State space exponentially larger
    \item Rotations in 4D space
\end{itemize}

\column{0.5\textwidth}
\centering
\textbf{New moves:}

\vspace{0.5em}

\begin{tabular}{ll}
\toprule
\textbf{Move} & \textbf{Meaning} \\
\midrule
FR & Front → Right \\
UO & Up → Outside \\
OR & Outside → Right \\
\bottomrule
\end{tabular}

\vspace{1em}

\textit{Two-letter notation for 4D rotations}
\end{columns}
\end{frame}

\begin{frame}{Research Question}
\begin{center}
\Large
\textcolor{blue}{How do repeated move sequences behave\\on a 4D hypercube?}

\vspace{2em}

\normalsize
\textbf{Key Concepts:}
\begin{itemize}
    \item \textbf{Orbit/Period}: Iterations to return to solved state
    \item \textbf{Chaos}: Sensitivity to perturbations (Lyapunov exponent)
    \item \textbf{Discrete Dynamics}: Iterating deterministic maps
\end{itemize}
\end{center}
\end{frame}

\section{Mathematical Framework}

\begin{frame}{Discrete Dynamical System}
\textbf{State Space:} $S$ = all possible puzzle configurations

\vspace{1em}

\textbf{Move Sequence:} $M = (m_1, m_2, \ldots, m_k)$

\vspace{1em}

\textbf{Composite Map:}
$$T_M: S \to S$$
$$T_M(s) = m_k \circ m_{k-1} \circ \cdots \circ m_1(s)$$

\vspace{1em}

\textbf{Trajectory:} Start at solved state $s_0$, iterate:
$$s_0 \xrightarrow{T_M} s_1 \xrightarrow{T_M} s_2 \xrightarrow{T_M} \cdots$$

\vspace{0.5em}

\textbf{Period $p$:} Minimum $n$ such that $T_M^n(s_0) = s_0$
\end{frame}

\begin{frame}{Lyapunov Exponents}
\textbf{Measuring Chaos:} How do perturbations grow?

\vspace{1em}

Given base sequence $M$ with period $p$, perturb it to $M'$:
\begin{itemize}
    \item Insert random move
    \item Remove a move
    \item Replace with different move
\end{itemize}

\vspace{1em}

Compute period $p'$ of perturbed sequence $M'$:
$$\lambda = \frac{1}{N} \sum_{i=1}^{N} \ln\left|\frac{p'_i}{p}\right|$$

\vspace{1em}

\textbf{Classification:}
\begin{itemize}
    \item $\lambda < 0.1$: \textcolor{blue}{Regular/Trivial}
    \item $0.1 \le \lambda < \ln(2)$: \textcolor{orange}{Sensitive}
    \item $\lambda \ge \ln(2) \approx 0.69$: \textcolor{red}{Chaotic}
\end{itemize}
\end{frame}

\section{Methods}

\begin{frame}{Computational Approach}
\textbf{Tool Stack:}
\begin{itemize}
    \item \texttt{ctrl/} - Rust trajectory analyzer (fast orbit detection)
    \item \texttt{obsv/} - Python statistical analysis (Lyapunov computation)
    \item \texttt{disp/} - Octave/MATLAB visualization
\end{itemize}

\vspace{1em}

\textbf{Systematic Testing:}
\begin{enumerate}
    \item Test all 64 two-move combinations (FR, UF, OR, ...)
    \item Compute periods using cycle detection (SHA256 state hashing)
    \item For interesting sequences: compute Lyapunov exponents
    \item Generate 10-20 perturbations per sequence
    \item Classify behavior: regular/sensitive/chaotic
\end{enumerate}

\vspace{1em}

\textbf{Puzzle:} 3×3×3×3 hypercube (via Hyperspeedcube library)
\end{frame}

\section{Results}

\begin{frame}{Key Findings}
\textbf{Single Moves:} All have period 8 (trivial, $\lambda = 0$)

\vspace{0.5em}

\textbf{Most Chaotic Sequences:}
\begin{center}
\begin{tabular}{llrr}
\toprule
\textbf{Sequence} & \textbf{Length} & \textbf{Period} & \textbf{$\lambda$} \\
\midrule
FR → FR & 2 & 4 & \textcolor{red}{5.94} \\
FO → FO & 2 & 4 & \textcolor{red}{\textbf{6.09}} \\
OF → OU → OB → OD & 4 & 6 & \textcolor{red}{5.64} \\
FR → OR → FL → OL & 4 & 12 & \textcolor{red}{4.65} \\
\midrule
FR → UF & 2 & 10,080 & \textcolor{red}{3.95} \\
FR → UO & 2 & 840 & \textcolor{red}{2.81} \\
\bottomrule
\end{tabular}
\end{center}

\vspace{0.5em}

\textbf{Surprising:} Self-compositions (FR→FR, FO→FO) are \textit{extremely} chaotic despite short periods!
\end{frame}

\begin{frame}{Period vs Chaos}
\begin{center}
\includegraphics[width=0.75\textwidth]{../disp/figures/period_vs_lambda.png}
\end{center}

\textbf{Key Insight:} Short periods $\ne$ simple behavior\\
Highly chaotic sequences can have very short orbits!
\end{frame}

\begin{frame}{Classification Distribution}
\begin{columns}
\column{0.5\textwidth}
\centering
\includegraphics[width=\textwidth]{../disp/figures/classification_pie.png}

\column{0.5\textwidth}
\textbf{Results (50 sequences):}

\vspace{1em}

\begin{itemize}
    \item \textcolor{blue}{Regular}: 15 sequences (30\%)
    \item \textcolor{orange}{Sensitive}: 12 sequences (24\%)
    \item \textcolor{red}{Chaotic}: 23 sequences (46\%)
\end{itemize}

\vspace{1em}

\textbf{Observation:} Nearly half of tested sequences exhibit chaotic behavior!
\end{columns}
\end{frame}

\begin{frame}{Top Chaotic Sequences}
\begin{center}
\includegraphics[width=0.85\textwidth]{../disp/figures/top_chaotic_sequences.png}
\end{center}

Self-compositions dominate the top rankings!
\end{frame}

\begin{frame}{Lyapunov vs Sequence Length}
\begin{center}
\includegraphics[width=0.75\textwidth]{../disp/figures/lyapunov_vs_length.png}
\end{center}

\textbf{Observation:} Highest chaos occurs at length 2 (self-compositions and pairs)
\end{frame}

\section{Visualizations}

\begin{frame}{Sequence Animations}
\begin{center}
\textbf{Animated GIFs available in \texttt{disp/figures/}}

\vspace{1em}

\begin{itemize}
    \item \texttt{sequence\_FR\_single.gif} - Baseline (Period 8, $\lambda = 0$)
    \item \texttt{sequence\_FO\_FO.gif} - \textcolor{red}{Most chaotic!} (Period 4, $\lambda = 6.09$)
    \item \texttt{sequence\_FR\_FR.gif} - Second most chaotic (Period 4, $\lambda = 5.94$)
\end{itemize}

\vspace{1em}

\textbf{Key Observations:}
\begin{itemize}
    \item Self-compositions create complex scrambling patterns
    \item Despite short periods (4 iterations), produce extreme chaos
    \item Visual inspection shows rapid state divergence
\end{itemize}

\vspace{0.5em}

\small{\textit{Note: GIFs show complete orbits (return to solved state)}}
\end{center}
\end{frame}

\section{Conclusions}

\begin{frame}{Key Takeaways}
\begin{enumerate}
    \item \textbf{4D hypercubes exhibit rich dynamics}
    \begin{itemize}
        \item 46\% of tested sequences are chaotic
        \item Periods range from 4 to 10,080
    \end{itemize}
    
    \vspace{0.5em}
    
    \item \textbf{Self-compositions are extremely chaotic}
    \begin{itemize}
        \item FR→FR, FO→FO have highest Lyapunov exponents ($\lambda > 5$)
        \item Despite having very short periods (4 iterations)
    \end{itemize}
    
    \vspace{0.5em}
    
    \item \textbf{Period $\ne$ complexity}
    \begin{itemize}
        \item Short orbits can be highly chaotic
        \item Long periods don't guarantee chaos
    \end{itemize}
    
    \vspace{0.5em}
    
    \item \textbf{Discrete chaos is real}
    \begin{itemize}
        \item Small perturbations cause massive orbit changes
        \item Lyapunov exponents successfully quantify sensitivity
    \end{itemize}
\end{enumerate}
\end{frame}

\begin{frame}{Future Directions}
\textbf{Theoretical:}
\begin{itemize}
    \item Why are self-compositions so chaotic?
    \item Connection to group theory structure?
    \item Predict chaotic sequences from move properties?
\end{itemize}

\vspace{1em}

\textbf{Computational:}
\begin{itemize}
    \item Test 5D+ hypercubes (if computationally feasible)
    \item Explore longer sequences (3-4+ moves)
    \item Investigate other perturbation types
\end{itemize}

\vspace{1em}

\textbf{Applications:}
\begin{itemize}
    \item Cryptographic pseudo-random generators?
    \item Physical systems with discrete symmetries?
\end{itemize}
\end{frame}

\begin{frame}{References}
\footnotesize

\textbf{Theory \& Background:}
\begin{itemize}
    \item Devaney, R. L. (2003). \textit{An Introduction to Chaotic Dynamical Systems}
    \item Joyner, D. (2008). \textit{Adventures in Group Theory: Rubik's Cube, Merlin's Machine, and Other Mathematical Toys}
    \item Rokicki, T. et al. (2014). \textit{The diameter of the Rubik's Cube group is twenty}
    \item Strogatz, S. H. (2015). \textit{Nonlinear Dynamics and Chaos}
\end{itemize}

\vspace{0.5em}

\textbf{Software \& Tools:}
\begin{itemize}
    \item \textbf{Hyperspeedcube} - HactarCE/Andrew J. Farkas\\
    \url{https://github.com/HactarCE/Hyperspeedcube}\\
    (MIT/Apache-2.0 License)
    \item \textbf{Rust}: clap, serde, sha2, hyperpuzzle ecosystem
    \item \textbf{Python}: NumPy, SciPy, Matplotlib, Pandas
    \item \textbf{Octave/MATLAB}: Visualization \& plotting
\end{itemize}

\end{frame}

\begin{frame}
\begin{center}
\Huge Thank You!

\vspace{2em}

\normalsize
\textbf{Questions?}

\vspace{2em}

\small
Code: \texttt{github.com/ltpie123/final}\\
Tools: Rust (ctrl), Python (obsv), Octave (disp)
\end{center}
\end{frame}

\end{document}
